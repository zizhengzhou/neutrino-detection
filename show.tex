\documentclass[aspectratio=169, 10pt]{beamer}

% Load the custom style file
\usepackage{UCAS}

%--------------------------------------------------------------------------
% Meta Information
%--------------------------------------------------------------------------
% \title{Proposal for Detector Shielding Design}
% \subtitle{Adoption of CONUS Experiment Geometry}
% \author[Your Name]{Your Name} % Short author in brackets, full name in braces
% \institute[UCAS]{University of Chinese Academy of Sciences}
% \date{\today}

\begin{document}

%--------------------------------------------------------------------------
% Title Slide
%--------------------------------------------------------------------------
% \begin{frame}
%     \titlepage
% \end{frame}

\section{CONUS}
\subsection{Geometry Proposal}
\begin{frame}{CONUS Shielding Geometry}
    \begin{columns}[c]
         
        \column{0.45\textwidth}
            10.1088/1742-6596/1342/1/012094
            \begin{figure}
                \centering 
                \includegraphics[width=.96\linewidth]{CONUS/geometry.png} 
                \caption{\footnotesize Top view of the shield layers.}
            \end{figure}
        \column{0.55\textwidth}
            \small
            \textbf{Layer Sequence (Outside $\to$ Inside):}
            \vspace{0.5em}
            
            \begin{itemize}
                \setlength{\itemsep}{0.15em} % Increase spacing between items
                
                \item \textbf{Outer Pb} (5 cm): \\
                Suppress external $\gamma$-radiation.  

                \item \textbf{Plastic Scintillator} (5.2 cm): \\
                Active veto for cosmic muons ($\mu$) (placed inside outer lead to reduce trigger rate).
                
                \item \textbf{Intermediate Pb} (2 $\times$ 5 cm): \\
                Shield radiation from veto PMTs.
                
                \item \textbf{Borated PE} (2 $\times$ 5 cm): \\
                Moderate fast neutrons and capture thermal neutrons.
                
                \item \textbf{Inner Pb} (2 $\times$ 5 cm): \\
                Absorb secondary $\gamma$-rays produced by neutron capture.
                
            \end{itemize}
            The sealed steel cage encloses the shield. 1.3 standard liters per minute breathing air purges the detector chamber to suppress Rn.
    \end{columns}
\end{frame}

\subsection{Background}
%--------------------------------------------------------------------------
% Slide 2: Shielding Effectiveness
%--------------------------------------------------------------------------
\begin{frame}{Suppression of Reactor-Correlated Background}
    \begin{columns}[c]
        
        % Left Column: Figure 27
        \column{0.5\textwidth}

        1903.09269
            \begin{figure}
                \centering
                \includegraphics[width=0.65\linewidth]{CONUS/component background event rates.png} 
                \caption{\footnotesize Nuclear recoil(Ge) backgrounds of neutrons.}
            \end{figure}

        % Right Column: Data Analysis
        \column{0.5\textwidth}
            \small
            \textbf{Key Findings from MC Simulation \& Measurement:}
            \vspace{0.8em}
            
            \begin{itemize}
                \setlength{\itemsep}{0.2em}
                
                \item \textbf{Reactor:}
                The neutron flux from the reactor core is \textbf{not leading} background after passing through the shield.
                
                \item \textbf{Muon includes:}
                Muon induced neutrons in shield should be mostly removed via veto  (97\%). 
                Muon induced neutrons in concrete (outside shield) is another neutron background in the same order.
                
                \item \textbf{Conclusion:} \\
                The proposed geometry (25 cm Pb + Borated PE) is \textbf{highly effective} against reactor radiation.
                The total background of neutrons in our ROI ($\sim 200$ events/y/kg) is smaller than the solar background.
                
            \end{itemize}
    \end{columns}
\end{frame}

\begin{frame}{Neutron flux(MC)}
    \begin{figure}
        \centering
        \includegraphics[width=0.7\textwidth]{CONUS/component background flux.png} 
    \end{figure}
\end{frame}

\begin{frame}{Gamma Ray Background Environment}
    \begin{columns}[T]
        \column{0.48\textwidth}
           1903.09269 Outside gamma rays
            \begin{figure}
                \centering
                \includegraphics[width=0.95\linewidth, height=0.36\textheight, keepaspectratio]{CONUS/gamma ray outside below 2700.png}
                \vspace{-0.3cm}
                \caption{Natural Radioactivity}
            \end{figure}
            
            \vspace{-0.7cm}
  
            \begin{figure}
                \centering
                \includegraphics[width=0.95\linewidth, height=0.36\textheight, keepaspectratio]{CONUS/gamma ray outside above 2700.png}
                \vspace{-0.3cm}
                \caption{Reactor-Correlated ${}^{16}\text{N}$ lines.}
            \end{figure}

        \column{0.52\textwidth}
            Gamma rays may have electron recoil (Compton scattering) and induce the background.
            \vspace{1em}

            \textbf{Shielding Efficiency (Internal):}
            \vspace{1em}
            \begin{itemize}
                \setlength{\itemsep}{1em}
                \item The \textbf{25 cm Pb} shield effectively blocks all gamma ray.
                \item \textbf{MC Simulation Result:} \\
                Inside the shield, the reactor-induced $\gamma$ count rate (0--450 keV) is:
                \begin{center}
                     $0.0401\pm0.0073$/kg/y
                \end{center}
                \item \textbf{Conclusion:} Gamma rays are \textbf{completely suppressed} and can be ignored.
            \end{itemize}
    \end{columns}
\end{frame}

\section{Spallation Neutron Source}

\begin{frame}
    \frametitle{COHERENT's Germanium Array}
     \begin{columns}[T]
        \column{0.48\textwidth}
           PhysRevLett.134.231801
            \begin{figure}
                \centering
                \includegraphics[width=0.95\linewidth]{spallation/geometry.png}
                \vspace{-0.2cm}
                \caption{Geometry}
            \end{figure}
             
        \column{0.52\textwidth}
            Technical details:
            \begin{itemize}
                \setlength{\itemsep}{1em}
                 \item Detector placed 19.2 m. $90^o$ degree  
                 \item SNS is 1.5 MW to 1.7 MW; 60 Hz. 29\% protons are converted to neutrinos.
                 \item Neutrons (including SNS induced) are shielded and ignored. 
                 \item HP Ge array ROI: $1.5-20 \rm\ keV_{ee} \approx 6.7-90\ keV_{nr}$.
                 \item Averaged neutrino flux is about $6\times 10^7 \rm cts/cm^2/s/flavor$ and peak neutrino flux is about $2\times 10^{12} \rm cts/cm^2/s/flavor$.
                 \item CENvS rate:$\sim 12.6 \rm cts /kg/y$, SNR: $\sim 1$.
            \end{itemize}
    \end{columns}
\end{frame}

\begin{frame}
    \frametitle{The result of COHERENT measurement}
        \begin{columns}
        \column{0.5\textwidth}
        \begin{figure}
            \centering
            \includegraphics[width=.75\textwidth]{spallation/on-off data.png}
        \end{figure}
        \column{0.5\textwidth}
        \begin{figure}
            \centering
            \includegraphics[width=\textwidth]{spallation/spectra.png}
        \end{figure}
    \end{columns}
    ...
\end{frame}

\begin{frame}
    \frametitle{Pulse creation time scale}
    \begin{columns}[T]
        \column{0.48\textwidth}
           10.1103/PhysRevD.106.032003
            \begin{figure}
                \centering
                \includegraphics[width=0.95\linewidth]{spallation/pulse time.png}
                \vspace{-0.2cm}
                \caption{Neutrino pulse creation time.}
            \end{figure}
             
        \column{0.52\textwidth}
        About the pulse shape:
            \begin{itemize}
                \setlength{\itemsep}{1em}
                \item Instant pulse (red) is from pion decay (26 ns) and thus its width depends on proton pulse. 
                \item Delay (blue) pulse from muon decay has a tail like $\exp(-\Gamma t)$ since muon life time is 2200ns.
            \end{itemize}
        \vspace{1em}

        For CSNS (170MW, 25Hz) averaged neutrino flux is $6.72 \times 10^6 \rm /cm^2/s$, and its peak neutrino flux is about $6\times 10^{11}\rm cts/cm^2/s/flavor$.

        \vspace{1em}

        The average flux domains the event rate and for our ROI (10-100eV) on CSNS, $<0.1$ events/year/kg is expected. 
        
        \textbf{COHERENT has an ROI 800 times wider and a source 10 times stronger.}
    \end{columns}
\end{frame}

\begin{frame}
    \frametitle{Event rates based on the averaged flux}
    \begin{columns}
        \column{0.5\textwidth}
        \begin{figure}
            \centering
            \includegraphics[width=\textwidth]{fig theory/CENvS/CSNS_nu_e.png}
        \end{figure}
        \column{0.5\textwidth}
        \begin{figure}
            \centering
            \includegraphics[width=\textwidth]{fig theory/CENvS/CSNS_anti_nu_mu.png}
        \end{figure}
    \end{columns}
    In range of nuclear recoil energy 0-10keV, the event rate per energy is nearly a constant and thus a wide ROI is helpful.
\end{frame}

\end{document}